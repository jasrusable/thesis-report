\documentclass[11pt,a4paper]{article}

\usepackage{graphicx}
\usepackage{amsmath}
\usepackage[backend=bibtex]{biblatex}
\addbibresource{sources.bib}

\title{Indoor positioning using the comparison of digital imagery to 3D models}
\date{2015}
\author{Jason David Russell}

\begin{document}

\maketitle
\thispagestyle{empty}

\pagenumbering{roman}
\setcounter{page}{0}

\newpage
\section*{Plagiarism Declaration}
	\addcontentsline{toc}{section}{Plagiarism Declaration}
	\begin{enumerate}
		\item
			I know that plagiarism means taking and using the ideas, writings, works or inventions of another as if they were one's own. I know that plagiarism not only includes verbatim copying, but also the extensive use of another person's ideas without proper acknowledgment (which includes the proper use of quotation marks). I know that plagiarism covers this sort of use of material found in textual sources and from the Internet.
		\item
			I acknowledge and understand that plagiarism is wrong.
		\item
			I understand that my research must be accurately referenced. I have followed the rules and conventions concerning referencing, citation and the use of quotations as set out in the Departmental Guide.
		\item
			This assignment is my own work, or my group's own unique group assignment.
			I acknowledge that copying someone else's assignment, or part of it, is wrong, and that submitting identical work to others constitutes a form of plagiarism.
		\item
			I have not allowed, nor will I in the future allow, anyone to copy my work with the intention of passing it off as their own work.
	\end{enumerate}
	Jason David Russell

\newpage
\section*{Acknowledgements}
	\addcontentsline{toc}{section}{Acknowledgements}
	I acknowledge everything.

\newpage
\section*{Abstract}
	\addcontentsline{toc}{section}{Abstract}
	This paper investigates the viability of a new indoor positioning system which is to make use of the comparason of digitial imagery to 3D models. An exaple scenario would be a person wishing to know which room they are in within a building. A 3D model of the building exists. The person takes a few photographs of her surroundings, the photographs are then compared to the 3D model and the result is presented to the person. The main focus of this paper is on the matching of the images to the 3D model.

\newpage
\tableofcontents

\pagenumbering{arabic}
\setcounter{page}{0}

\newpage
\section{Introduction}
	\subsection{Subject of the Report}
		This report concerns the investigation of a proposed alternative indoor positioning method. This proposed method makes use of the comparson of digital imagery to 3D models in order to determine position.
	
	\subsection{Background to the Report}
		The background to the report should go here.
	
	\subsection{Objectives of the Report}
		The objectives of this report are therefore to:
		\begin{itemize}
			\item
				discuss related work
			\item
				propose the alternate method
			\item
				present and discuss experimental results
			\item
				 consider future work
			\item
				draw conclusions
		\end{itemize}
	
	\subsection{Scope and limitaions}
		Some scope and limitaion stuff goes here.
	
	\subsection{Plan of Development}
		This report begins by looking at work related to indoor positioning systems. Following on from that, the alternate method will be proposed. Thereafter results of investigations will be presented and discussed.

\newpage
\section{Related Work}
	In the sections to follow, exsiting indoor positioning systems will be discussed. Two main branches of technologies will be covered, namely radio and non-radio technologies.
	
	\subsection{Radio technologies}
		Any wireless technology can in theory, be used for positioning. Three main techniques exist, each one will be briefly outlined below. The sections to follow will look at exisiting systems which employ these techniques.
	
	\paragraph{Time of arrival}
		Time of arrival (or time of flight) is the travel time of a radio signal from a single transmitter to a single remote reciever. The basic observable is time. A Distance can be directly calculated using the known propogation velocity of signals with the basic observable time. Location can then be determined using multilateration. This requires a setup of at least one reciver and three transmiters or vise versa.
		Required with such systems is the synchronization of clocks between recievers and transmitters.
		Time of travel systems suffer greatly from effects such as multipath.
		\cite{k._pahlavan_wideband_1998}
	
	\paragraph{Recieved signal strength indication}
		Recieved signal strength indication is a measure of the power level recieved by a sensor. Because of how radio waves propogate, distance can be approximated between a transmitter and reciever based on the relationship between the transmitted and recieved signal strength. Multilateration can be used to determine locality of a sensor.
	
	\paragraph{Angle of arival}
		Angle of arrival is the angle from which a signal is recieved at a reciever. Angle of arrival is typically determined by using the time difference of arrival between multiple antennas in a sensor array. Position is determined using multiangulation.
		Angle of arival systems also suffer from effects such as multipath.
	
	\subsubsection{Wi-Fi based systems}
		Wi-Fi based technologies are based on measurements such as TOA (time of arrival), TDOA (time difference of arrival) and DOA (direction of arrival). However these techniques are severly impared when line of sight is not achievable. These techniques also suffer due to objects such as walls and floors and other objects which attenuate and reflect the signals, directly affecting accuracies.
	
		An alternavte Wi-Fi based geolocation method has been explored which makes use of relative signal strengths bewteen transmitters and recievers. So instead of measuring the time or angle of signals, the signal strengh is used to determine location. This negatates many of the above mentioned deficiencies. A similar system has been employed by animal trackers with directional antennas.
		\cite{yongguang_chen_signal_2002}
	
	\subsubsection{Bluetooth systems}
		Bluetooth positioning works by using proximities, as apposed to angulation or lateration. As such, exact locations are not attainable using these techniques. Instead, the system acts more as a geofence and works by determining which node a device is currently connected to in order to determine the location of the device.
	
		Apple has developed a protocol called iBeacon. This porotocol makes use fo Bluetooth proximity techniques in order to enable smart devices to perform actions when in close proximity to iBeacon.
		\cite{_everything_????}
	
	\subsubsection{Choke point concepts}
		Choke point systems work by locating and indexing tagged objects in order to track them. The concept works by passing tagged objects though a choke point (or gate), the choke point will then have a sensor which detects the tagged object passing through the gate. Many chocke point sensors work with passive radio-frequency identification (RFID) tags which do not report distances or signal strengths.
		\cite{reza_investigation_2008}
	
	\subsubsection{Grid concepts}
		Grid concepts employ a dense network of low-range reciervers arranged in a known pattern. A tagged object will be sensed by only a few nearby, networks recievers. By determining which recievers and tag is sensed by, a rough approximation of the location of the tagged object can be made.
	
	\subsubsection{Others}
		Various other systems exist but will not be discussed further. Thses include ultra-wide band (UWB), infared (IR), visible light communication, and ultrasound.
	
	\subsection{Non-radio technologies}
		Non-radio technologies which can be used for indoor positioning will be discussed below. These systems can provide increased accuracy at the expense of increased costs of equipment and additional installations.
	
	\subsubsection{Magnetic positioning}
		Magnetic positioning takes advantage of the way iron in buildings affects the Earths magnetic field. The iron in buildings creates local variations in the Earths magnetic field which can then be sensed by compases to map indoor locations.
		\cite{supreeth_sudhakaran_geospatial_2014}
	
	\subsubsection{Inertial measurements}
		Inertial measurement units (IMUs) can be carried by an object in order to track the objects path through space. IMUs measure acceleration and orientaion along three orthogonal axis using accelerometers and gyroscopes. Position can be determined by double integration of the acceleration measurements - this is a form of dead reckoning. Dead reckoning is the process of calculating ones current location by using previously determined positions with esimated speed and orientation over some time. This yields relative position estimations. Dead reckoning is subject to what is known as drift which is an accumulation of errors. Due to the succeptability of IMUs to drift, they are often used in conjunction with other positioning systems in order to correct for this drift.

\newpage
\section{Proposed method}
	\subsection{The 3D model}
		\subsubsection{Data acqusition}
			Direct measurements
				Tape measure, theodolite, plans etc
			Point clouds
				I used a point cloud
				From laser scanning
				Resolution/precision of the point cloud
				Cleaning up the point cloud
				Standardizing the point cloud (uniform resolution)
		\subsubsection{Model creation}
			Automated techniques failed
			Model created with Blender + point cloud
			Discuss level of detail
				Elements (degree of detail)
					Include conduate, wall plugs, whiteboard seams etc
				Lighting/shadows
					Position of lights and shadows, type of lights
				Texture
					Rough texture for carpet, smooth texture for glass panel on door
				Image draping
					Image drapped over air vent
					
				

\section{Experimental implementaion}

\section{Discussions}

\section{Conclusions and future work}

\newpage
\printbibliography

\end{document}